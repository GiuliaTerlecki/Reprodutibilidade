% Options for packages loaded elsewhere
\PassOptionsToPackage{unicode}{hyperref}
\PassOptionsToPackage{hyphens}{url}
%
\documentclass[
  preprint, 3p, authoryear]{article}
\usepackage{amsmath,amssymb}
\usepackage{iftex}
\ifPDFTeX
  \usepackage[T1]{fontenc}
  \usepackage[utf8]{inputenc}
  \usepackage{textcomp} % provide euro and other symbols
\else % if luatex or xetex
  \usepackage{unicode-math} % this also loads fontspec
  \defaultfontfeatures{Scale=MatchLowercase}
  \defaultfontfeatures[\rmfamily]{Ligatures=TeX,Scale=1}
\fi
\usepackage{lmodern}
\ifPDFTeX\else
  % xetex/luatex font selection
\fi
% Use upquote if available, for straight quotes in verbatim environments
\IfFileExists{upquote.sty}{\usepackage{upquote}}{}
\IfFileExists{microtype.sty}{% use microtype if available
  \usepackage[]{microtype}
  \UseMicrotypeSet[protrusion]{basicmath} % disable protrusion for tt fonts
}{}
\makeatletter
\@ifundefined{KOMAClassName}{% if non-KOMA class
  \IfFileExists{parskip.sty}{%
    \usepackage{parskip}
  }{% else
    \setlength{\parindent}{0pt}
    \setlength{\parskip}{6pt plus 2pt minus 1pt}}
}{% if KOMA class
  \KOMAoptions{parskip=half}}
\makeatother
\usepackage{xcolor}
\usepackage[margin=1in]{geometry}
\usepackage{graphicx}
\makeatletter
\def\maxwidth{\ifdim\Gin@nat@width>\linewidth\linewidth\else\Gin@nat@width\fi}
\def\maxheight{\ifdim\Gin@nat@height>\textheight\textheight\else\Gin@nat@height\fi}
\makeatother
% Scale images if necessary, so that they will not overflow the page
% margins by default, and it is still possible to overwrite the defaults
% using explicit options in \includegraphics[width, height, ...]{}
\setkeys{Gin}{width=\maxwidth,height=\maxheight,keepaspectratio}
% Set default figure placement to htbp
\makeatletter
\def\fps@figure{htbp}
\makeatother
\setlength{\emergencystretch}{3em} % prevent overfull lines
\providecommand{\tightlist}{%
  \setlength{\itemsep}{0pt}\setlength{\parskip}{0pt}}
\setcounter{secnumdepth}{5}
\ifLuaTeX
  \usepackage{selnolig}  % disable illegal ligatures
\fi
\usepackage{bookmark}
\IfFileExists{xurl.sty}{\usepackage{xurl}}{} % add URL line breaks if available
\urlstyle{same}
\hypersetup{
  pdftitle={Seasonal patterns on isotopic niches and diet of Bigeye and Southern Spotted Opah (Lamprididae) in Southwestern Atlantic Ocean},
  pdfkeywords={Opah, isotopic niche, diet, Atlantic Ocean},
  hidelinks,
  pdfcreator={LaTeX via pandoc}}

\title{Seasonal patterns on isotopic niches and diet of Bigeye and
Southern Spotted Opah (Lamprididae) in Southwestern Atlantic Ocean}
\author{true \and true \and true}
\date{2024-09-11}

\begin{document}
\maketitle
\begin{abstract}
Abstract

Opahs (Lampsis spp.) are large deep-water epi-mesopelagic predator
fishes captured worldwide as bycatch of longline fisheries targeting
large pelagic fishes. Despite the growing culinary interest leading to
increasing commercial interest, several basic biological information
about the species is still poorly known. This study uses stable isotope
and stomach content analysis to access the diet and seasonality on the
isotopic niche of the Big-eye Opah, Lampris megalopsis, and the Southern
Spotted Opah Lampris australensis in the Southwest Atlantic Ocean
(SWAO). Generalized Linear Models were applied to investigate the
influence of the species, sex, seasons, and furcal length on δ13C and
δ15N isotopic compositions. Significant differences were observed only
for Autumn and for L. megalopsis. The isotopic niches resulted in
overlapped 40\% ellipses between the species. Seasonal differences for
δ15N in hot and cold seasons for both species related to the dynamic of
the Brazilian and the Malvinas (Falkland) currents and the shift on the
baseline source of nitrogen. Differences in δ13C, with enriched values
in the warmer season, were observed only for L. megalopsis and suggested
movements to areas with depleted 15C values. Diet for both species was
composed predominantly by Cephalopods and Teleost's, followed by
Crustacea, in smaller quantities. An alarming high plastic frequency of
occurrence was observed in 40\% of the stomachs of L. megalopsis and
31\% of L. australensis. This study advances in understanding the Opah
fishes feeding ecology in SWAO and provides information on community
dynamics and the functional role that these species play in the
structure of all marine ecosystems where they occur. Given the growing
global commercial importance of Lampris spp., it is also increasingly
important to know their inter and intraspecific relationships and the
anthropological impacts they are suffering.
\end{abstract}



\end{document}
