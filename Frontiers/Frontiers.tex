% Options for packages loaded elsewhere
\PassOptionsToPackage{unicode}{hyperref}
\PassOptionsToPackage{hyphens}{url}
%
\documentclass[
  utf8]{FrontiersinHarvard}
\usepackage{amsmath,amssymb}
\usepackage{iftex}
\ifPDFTeX
  \usepackage[T1]{fontenc}
  \usepackage[utf8]{inputenc}
  \usepackage{textcomp} % provide euro and other symbols
\else % if luatex or xetex
  \usepackage{unicode-math} % this also loads fontspec
  \defaultfontfeatures{Scale=MatchLowercase}
  \defaultfontfeatures[\rmfamily]{Ligatures=TeX,Scale=1}
\fi
\usepackage{lmodern}
\ifPDFTeX\else
  % xetex/luatex font selection
\fi
% Use upquote if available, for straight quotes in verbatim environments
\IfFileExists{upquote.sty}{\usepackage{upquote}}{}
\IfFileExists{microtype.sty}{% use microtype if available
  \usepackage[]{microtype}
  \UseMicrotypeSet[protrusion]{basicmath} % disable protrusion for tt fonts
}{}
\makeatletter
\@ifundefined{KOMAClassName}{% if non-KOMA class
  \IfFileExists{parskip.sty}{%
    \usepackage{parskip}
  }{% else
    \setlength{\parindent}{0pt}
    \setlength{\parskip}{6pt plus 2pt minus 1pt}}
}{% if KOMA class
  \KOMAoptions{parskip=half}}
\makeatother
\usepackage{xcolor}
\usepackage{color}
\usepackage{fancyvrb}
\newcommand{\VerbBar}{|}
\newcommand{\VERB}{\Verb[commandchars=\\\{\}]}
\DefineVerbatimEnvironment{Highlighting}{Verbatim}{commandchars=\\\{\}}
% Add ',fontsize=\small' for more characters per line
\usepackage{framed}
\definecolor{shadecolor}{RGB}{248,248,248}
\newenvironment{Shaded}{\begin{snugshade}}{\end{snugshade}}
\newcommand{\AlertTok}[1]{\textcolor[rgb]{0.94,0.16,0.16}{#1}}
\newcommand{\AnnotationTok}[1]{\textcolor[rgb]{0.56,0.35,0.01}{\textbf{\textit{#1}}}}
\newcommand{\AttributeTok}[1]{\textcolor[rgb]{0.13,0.29,0.53}{#1}}
\newcommand{\BaseNTok}[1]{\textcolor[rgb]{0.00,0.00,0.81}{#1}}
\newcommand{\BuiltInTok}[1]{#1}
\newcommand{\CharTok}[1]{\textcolor[rgb]{0.31,0.60,0.02}{#1}}
\newcommand{\CommentTok}[1]{\textcolor[rgb]{0.56,0.35,0.01}{\textit{#1}}}
\newcommand{\CommentVarTok}[1]{\textcolor[rgb]{0.56,0.35,0.01}{\textbf{\textit{#1}}}}
\newcommand{\ConstantTok}[1]{\textcolor[rgb]{0.56,0.35,0.01}{#1}}
\newcommand{\ControlFlowTok}[1]{\textcolor[rgb]{0.13,0.29,0.53}{\textbf{#1}}}
\newcommand{\DataTypeTok}[1]{\textcolor[rgb]{0.13,0.29,0.53}{#1}}
\newcommand{\DecValTok}[1]{\textcolor[rgb]{0.00,0.00,0.81}{#1}}
\newcommand{\DocumentationTok}[1]{\textcolor[rgb]{0.56,0.35,0.01}{\textbf{\textit{#1}}}}
\newcommand{\ErrorTok}[1]{\textcolor[rgb]{0.64,0.00,0.00}{\textbf{#1}}}
\newcommand{\ExtensionTok}[1]{#1}
\newcommand{\FloatTok}[1]{\textcolor[rgb]{0.00,0.00,0.81}{#1}}
\newcommand{\FunctionTok}[1]{\textcolor[rgb]{0.13,0.29,0.53}{\textbf{#1}}}
\newcommand{\ImportTok}[1]{#1}
\newcommand{\InformationTok}[1]{\textcolor[rgb]{0.56,0.35,0.01}{\textbf{\textit{#1}}}}
\newcommand{\KeywordTok}[1]{\textcolor[rgb]{0.13,0.29,0.53}{\textbf{#1}}}
\newcommand{\NormalTok}[1]{#1}
\newcommand{\OperatorTok}[1]{\textcolor[rgb]{0.81,0.36,0.00}{\textbf{#1}}}
\newcommand{\OtherTok}[1]{\textcolor[rgb]{0.56,0.35,0.01}{#1}}
\newcommand{\PreprocessorTok}[1]{\textcolor[rgb]{0.56,0.35,0.01}{\textit{#1}}}
\newcommand{\RegionMarkerTok}[1]{#1}
\newcommand{\SpecialCharTok}[1]{\textcolor[rgb]{0.81,0.36,0.00}{\textbf{#1}}}
\newcommand{\SpecialStringTok}[1]{\textcolor[rgb]{0.31,0.60,0.02}{#1}}
\newcommand{\StringTok}[1]{\textcolor[rgb]{0.31,0.60,0.02}{#1}}
\newcommand{\VariableTok}[1]{\textcolor[rgb]{0.00,0.00,0.00}{#1}}
\newcommand{\VerbatimStringTok}[1]{\textcolor[rgb]{0.31,0.60,0.02}{#1}}
\newcommand{\WarningTok}[1]{\textcolor[rgb]{0.56,0.35,0.01}{\textbf{\textit{#1}}}}
\usepackage{graphicx}
\makeatletter
\def\maxwidth{\ifdim\Gin@nat@width>\linewidth\linewidth\else\Gin@nat@width\fi}
\def\maxheight{\ifdim\Gin@nat@height>\textheight\textheight\else\Gin@nat@height\fi}
\makeatother
% Scale images if necessary, so that they will not overflow the page
% margins by default, and it is still possible to overwrite the defaults
% using explicit options in \includegraphics[width, height, ...]{}
\setkeys{Gin}{width=\maxwidth,height=\maxheight,keepaspectratio}
% Set default figure placement to htbp
\makeatletter
\def\fps@figure{htbp}
\makeatother
\setlength{\emergencystretch}{3em} % prevent overfull lines
\providecommand{\tightlist}{%
  \setlength{\itemsep}{0pt}\setlength{\parskip}{0pt}}
\setcounter{secnumdepth}{-\maxdimen} % remove section numbering
\ifLuaTeX
  \usepackage{selnolig}  % disable illegal ligatures
\fi
\usepackage{bookmark}
\IfFileExists{xurl.sty}{\usepackage{xurl}}{} % add URL line breaks if available
\urlstyle{same}
\hypersetup{
  pdftitle={Uso de Habitat de Tubarões Azul e Anequim no Oceano Atlântico Sul},
  hidelinks,
  pdfcreator={LaTeX via pandoc}}

\title{Uso de Habitat de Tubarões Azul e Anequim no Oceano Atlântico
Sul}
\author{true \and true \and true}
\date{}

\begin{document}
\maketitle

\section*{Abstract}\label{abstract}
\addcontentsline{toc}{section}{Abstract}

Este estudo analisa o uso de habitat, a sobreposição de nicho e a
conectividade do tubarão azul (\emph{Prionace glauca}) e do anequim
(\emph{Isurus oxyrinchus}) no Atlântico Sul. Utilizando análises
isotópicas e microquímicas em vértebras, investigamos mudanças
ontogenéticas no uso de habitat e comportamentos alimentares dessas
espécies. Os objetivos incluem avaliar a sobreposição de nicho,
identificar especializações tróficas e verificar a conectividade do
tubarão azul entre regiões do Atlântico Sul. Esperamos demonstrar uma
sobreposição de nicho entre as espécies e evidências de conectividade
transoceânica para o tubarão azul, fornecendo dados essenciais para
manejo ecossistêmico e conservação.

keywords: - Conectividade - Sobreposição de nicho - Tubarão azul -
Atlântico Sul - Análise isotópica - Ecologia trófica - Migração
transoceânica - Gestão ecossistêmica

output: rticles::frontiers\_article bibliography: test.bib biblio-style:
Frontiers-Harvard ---

\section*{Introduction}\label{introduction}
\addcontentsline{toc}{section}{Introduction}

O tubarão azul (\emph{Prionace glauca}) e o anequim (\emph{Isurus
oxyrinchus}) são duas das principais espécies de tubarões pelágicos do
Atlântico, com distribuições amplas e migratórias que as tornam
vulneráveis à exploração pesqueira {[}@Campana2004; @Stevens2010{]}.
Essas espécies desempenham papéis ecológicos fundamentais nos
ecossistemas marinhos, regulando as populações de suas presas e
contribuindo para a manutenção do equilíbrio ecológico
{[}@Heithaus2008{]}. Estudos recentes destacam a importância de
compreender a conectividade populacional desses tubarões para
desenvolver estratégias de manejo que considerem sua natureza
transoceânica e evitem o declínio populacional {[}@Queiroz2019;
@Ferreira2017{]}.

O uso de análises isotópicas e microquímicas em vértebras de tubarões
tem se mostrado uma ferramenta valiosa para rastrear padrões de
migração, conectividade e mudanças ontogenéticas em hábitos alimentares
{[}@Estrada2006; @Hussey2015{]}. Através dessas análises, é possível
identificar as áreas de alimentação e os movimentos sazonais das
espécies, fornecendo insights sobre as dinâmicas de habitat e as
interações tróficas ao longo de diferentes fases da vida dos tubarões
{[}@Carlisle2015; @MacNeil2005{]}.

\subsubsection*{Fórmula para Posição
Trófica}\label{fuxf3rmula-para-posiuxe7uxe3o-truxf3fica}
\addcontentsline{toc}{subsubsection}{Fórmula para Posição Trófica}

Para estimar a posição trófica dos tubarões analisados, foi utilizada a
seguinte fórmula de enriquecimento isotópico de nitrogênio entre
predador e presa:

\[
\Delta^{15}N = \delta^{15}N_{predador} - \delta^{15}N_{presa}
\]

onde \(\delta^{15}N\) representa o valor isotópico de nitrogênio nas
amostras de vértebras, possibilitando inferir a posição trófica e os
hábitos alimentares ontogenéticos dos indivíduos {[}@Shiffman2019;
@Rooker2008{]}.

Estudos como o de @Queiroz2019 demonstram que o tubarão azul exibe
conectividade entre regiões do Atlântico, reforçando a necessidade de
uma abordagem de manejo cooperativa entre diferentes países. Este estudo
visa investigar a sobreposição de nicho entre o tubarão azul e o anequim
no Atlântico Sul e avaliar a conectividade do tubarão azul entre as
regiões sudoeste e sudeste do oceano.

\section*{Results}\label{results}
\addcontentsline{toc}{section}{Results}

Espera-se que os resultados deste estudo revelem uma sobreposição
significativa de nicho entre o tubarão azul (\emph{Prionace glauca}) e o
anequim (\emph{Isurus oxyrinchus}) no Atlântico Sul. A análise isotópica
deverá indicar variações na posição trófica entre as fases ontogenéticas
dos tubarões, evidenciando especializações alimentares em diferentes
estágios de vida {[}@Carlisle2015; @Estrada2006{]}.

\subsubsection*{Sugestão de Imagem}\label{sugestuxe3o-de-imagem}
\addcontentsline{toc}{subsubsection}{Sugestão de Imagem}

Um \textbf{mapa do Atlântico Sul} com as rotas de migração estimadas dos
tubarões azul e anequim, destacando os pontos de coleta das amostras de
vértebras, é uma adição importante. Esse mapa visualizaria as conexões
entre as regiões sudoeste e sudeste do Atlântico e outras áreas
relevantes, demonstrando visualmente a distribuição e possíveis rotas
migratórias das duas espécies.

\begin{Shaded}
\begin{Highlighting}[]
\CommentTok{\# Simulação de dados}
\FunctionTok{set.seed}\NormalTok{(}\DecValTok{999}\NormalTok{)}
\NormalTok{idades }\OtherTok{\textless{}{-}} \DecValTok{1}\SpecialCharTok{:}\DecValTok{20}  \CommentTok{\# Classes etárias de 1 a 20 anos}
\NormalTok{trop\_shark }\OtherTok{\textless{}{-}} \DecValTok{3} \SpecialCharTok{+} \FloatTok{0.1} \SpecialCharTok{*}\NormalTok{ idades }\SpecialCharTok{+} \FunctionTok{rnorm}\NormalTok{(}\FunctionTok{length}\NormalTok{(idades), }\AttributeTok{sd =} \FloatTok{0.3}\NormalTok{)  }\CommentTok{\# Posição trófica simulada do tubarão azul}
\NormalTok{trop\_mako }\OtherTok{\textless{}{-}} \FloatTok{3.5} \SpecialCharTok{+} \FloatTok{0.08} \SpecialCharTok{*}\NormalTok{ idades }\SpecialCharTok{+} \FunctionTok{rnorm}\NormalTok{(}\FunctionTok{length}\NormalTok{(idades), }\AttributeTok{sd =} \FloatTok{0.3}\NormalTok{)  }\CommentTok{\# Posição trófica simulada do anequim}

\CommentTok{\# Plotando o gráfico}
\FunctionTok{plot}\NormalTok{(idades, trop\_shark, }\AttributeTok{type =} \StringTok{"o"}\NormalTok{, }\AttributeTok{col =} \StringTok{"blue"}\NormalTok{, }\AttributeTok{ylim =} \FunctionTok{c}\NormalTok{(}\DecValTok{3}\NormalTok{, }\DecValTok{6}\NormalTok{),}
     \AttributeTok{main =} \StringTok{"Posição Trófica por Idade para Tubarões Azul e Anequim"}\NormalTok{,}
     \AttributeTok{xlab =} \StringTok{"Idade (anos)"}\NormalTok{, }\AttributeTok{ylab =} \StringTok{"Posição Trófica"}\NormalTok{)}
\FunctionTok{lines}\NormalTok{(idades, trop\_mako, }\AttributeTok{type =} \StringTok{"o"}\NormalTok{, }\AttributeTok{col =} \StringTok{"red"}\NormalTok{)}
\FunctionTok{legend}\NormalTok{(}\StringTok{"topleft"}\NormalTok{, }\AttributeTok{legend =} \FunctionTok{c}\NormalTok{(}\StringTok{"Tubarão Azul"}\NormalTok{, }\StringTok{"Anequim"}\NormalTok{), }\AttributeTok{col =} \FunctionTok{c}\NormalTok{(}\StringTok{"blue"}\NormalTok{, }\StringTok{"red"}\NormalTok{), }\AttributeTok{lty =} \DecValTok{1}\NormalTok{, }\AttributeTok{pch =} \DecValTok{1}\NormalTok{)}
\end{Highlighting}
\end{Shaded}

\includegraphics{Frontiers_files/figure-latex/trophic_graph-1.pdf}

\end{document}
